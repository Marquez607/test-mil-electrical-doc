
\documentclass[a4paper,12pt,oneside,pdflatex,italian,final,twocolumn]{article}



\usepackage[utf8]{inputenc}
\usepackage{parallel}
\usepackage{siunitx}
\usepackage{booktabs}
\usepackage{fancyhdr}

\usepackage[export]{adjustbox}
\usepackage[margin=0.5in]{geometry}
\addtolength{\topmargin}{0in}

\usepackage{libertine}
\renewcommand*\familydefault{\sfdefault}  %% Only if the base font of the document is to be sans serif
\usepackage[T1]{fontenc}

\title{BSPD 2019}
\author{arsphotographika }
\date{April 2019}

\begin{document}

\pagestyle{fancy}

\lhead{Machine Intelligence Lab}
\chead {Thruster/Kill System}
\rhead{MIL0012-RA}


\onecolumn

\begin{figure}
\begin{minipage}{0.47\textwidth}
\centering
\includegraphics[width=.7\textwidth,left,]{MIL_LOGO.png}

\end{minipage}
\hfill
\begin{minipage}{0.47\textwidth}
\raggedleft
\Huge \textbf{MIL0012-RA}
\Huge \textbf{Thruster/Kill System}
\end{minipage}
\end{figure}


\begin{figure}
\begin{minipage}{0.47\textwidth}

\section{Electrical Team Members}

        Firmware Des. By: Marquez Jones \\
        PCB Des. By: Frank Mitchell\\
  


\end{minipage}
\hfill
\begin{minipage}{0.47\textwidth}
\centering
\includegraphics[width=0.7\textwidth,right]{TKB_Board.PNG}

\end{minipage}
\end{figure}


\vspace{-8cm}
\section{Description}
\vspace{-5cm}
    The Thruster/Kill System describes the PCB and accompanying Firmware to control and monitor Blue Robotics Thrusters and Kill network on Subjugator 8. The hardware for this system is designed to interface with the vehicle's CAN communication system, 8 Blue robotics ESCs via PWM, and control power relays to both main vessel and its thrusters. A note about this, is the main vessel relay connection was repurposed in firmware to accommodate another relay as opposed to being an optional UART connection as designated in hardware. The CAN handling on this board is designed to filter for motherboard commands on the thruster channel and read in all kill messages sent on the kill channel. The board will also output the status of a hall effect sensor corresponding to GO periodically(explained later). The board also has two other hall effect connections which correspond to Hard Kill(On/Off) and Soft Kill(Enable). If a Hard Kill is asserted, power to main will be cut. 
    If a Soft Kill is asserted via either CAN or the Hall effect, power to thrusters will be cut. Other possible sources of kill are documented in this manual. 


\pagebreak
\raggedright
\section{Firmware FSM}
\vspace{-5cm}
\raggedright
\subsection{Main Program}
\vspace{-15cm}
\centering
\begin{figure}[h]
\centering
\includegraphics[width=1\textwidth,]{MAIN_FSM_R1.PNG}
\end{figure}

\pagebreak
\raggedright
\subsection{Interrupt Service Routines}
\vspace{-10cm}
\centering
\begin{figure}[h]
\centering
\includegraphics[width=1\textwidth,]{ISRS_FSM_R1.PNG}
\end{figure}


\pagebreak

\raggedright
\section{Kill System}
\vspace{-1cm}
\subsection{Overview}
\vspace{-1cm}
THe Kill System describes sub's safety features that prevents that disables the system thus disabling its ability to act erratically. There's multiple reasons as to why we may want to kill sub and there are several ways to accomplish killing sub. Sub has two external ways to kill via hall effect magnets that should be mounted to the bulkhead. There's also the ability to soft kill sub via a CAN message. Two passive kill features also exist,one being losing communication with motherboard and the other is a softer idle where it'll tell thrusters to stop if motherboard ceases sending thruster commands. Refer to the table below to see kill sources and their effects. 

\vspace{-4cm}
\subsection{Kill Sources and Effects}
\centering
\begin{tabular}{c|c|c|c}
\toprule
Sources             & Kills Thruster Power & Kills Main Power  & Locks Program  \\
\midrule
Hall Hard Kill      & Yes                 & Yes               &  Yes\\
Hall Soft Kill      & Yes                 & No                &  Yes\\
CAN  Soft Kill      & Yes                 & No                &  Yes\\
Heart Beat Missed   & Yes                 & No                &  No \\
Thruster Idle       & No                  & No                &  No \\
\bottomrule
\end{tabular}

\vspace{-5cm}
\raggedright
\subsection{Passive Kill Features}
\vspace{-1cm}
This section will more in depth describe passive or kills that aren't caused by outside request or action. These are kills the board will enact if the board believes motherboard is inactive. The two passive kill features are the Heart Beat Kill and Thruster Idle. These are timing sensitive events. 
Idles will occur every 500 ms. Heart Beat Kills will occur every 1000 ms. 

\vspace{-4cm}
\subsection{Passive Kill Table}
\vspace{-1cm}
\centering
\begin{tabular}{c|c|c}
\toprule
Sources             & Period              & To Unkill  \\
\midrule
Heart Beat Missed   & 1000ms                 & Transmit Heart Beat Message               \\ 
Thruster Idle       & 500ms                  & Transmit Thruster Command               \\
\bottomrule
\end{tabular}



\pagebreak
\raggedright
\section{Thruster System}
\vspace{-2cm}
\centering
\begin{tabular}{c|c}
\toprule
ID & Thruster \\             
\midrule
0 & FHL \\
1 & FHR \\
2 & FVL \\
3 & FVR \\
4 & BHL \\
5 & BHR \\ 
6 & BVL \\
7 & BVR \\
\bottomrule
\end{tabular}

\vspace{-12cm}
\begin{itemize}
\item The firmware on this board used the TM4C123G PWM module to communicate with the Blue Robotics Basic ESCs as opposed to providing PWM commands to the thrusters themselves. In this, I used the word communicate. This is because the ESCs accept a protocol similar to One Shot which effectively uses PWM to send commands. This is abstracted in CAN interface as the board will receive a thrust between 1 and -1 then convert that into the correct command to the ESCs. In this, it sends a PWM signal with a 2000us period. This is important to know if you wish to modify the firmware or if we want to use the board to control thrusters that aren't blue robotics. \\
For this board, each thruster is assigned an ID 0 to 7.This is how commands are distributed. This will make more sense when looking at the thruster command packet. The thruster mapping is posted here. Note this is based on the connections diagram. \\
\end{itemize}

\pagebreak
\raggedright
\section{Communications}
\begin{itemize}
\item In this version of the firmware, the primary means to communicate with the board is via control area network(CAN). This board also falls in line with the MIL CAN protocol of using task groups/channels. The firmware is set up to receive messages from both the thruster and kill channels. Check the ID assignment spreadsheets for the specifics on what ECUs and ECU IDs these refer to. In regards to ID filtering, this board will receive any kill messages on the kill channel. In this, if a Kill command is sent on the channe, it will kill. This board also filters for motherboard specific messages on the Thruster channel. \\
\vspace{-1cm}
\item This board will transmit two different messages. It will periodically transmit the status of the Go hall effect sensor. The only other case it should be transmitting is if the board is either soft or hard killed \\
\end{itemize}

\vspace{-20cm}
\centering
\begin{tabular}{c|c|c}
\toprule
Hex & Ascii & Meaning \\             
\midrule
0x54 & 'T' &  Thruster Header Byte\\
0x4B & 'K' &  Kill Header Byte\\
0x47 & 'G'&   Go Header Byte\\
0x48 & 'H' &  Heart Beat Header Byte\\
0x43 & 'C' &  Command\\
0x52 & 'R' &  Response\\ 
0x41 & 'A' &  Assert\\
0x55 & 'U' &  Unassert\\
0x48 & 'H' &  Hard Kill\\
0x53 & 'S' &  Soft Kill\\
0x4D & 'M' &  Mother Board\\
\bottomrule
\end{tabular}



\pagebreak
\raggedright
\subsection{Kill Command/Response Message(RX/TX)}
Channel = Kill Channel\\ 
Will sending a 'C'(command) will tell the board to instantiate a kill by which the board will\\respond with 'R' to verify its state. 
\centering
\begin{figure}[h]
\centering
\includegraphics[width=1\textwidth,]{Kill_Msg.PNG}
\end{figure}

\raggedright
\subsection{Thruster Command Message(RX Only)}
\raggedright
Channel = Thruster Channel\\ 
Is used to transmit thruster commands.
Float value should be between -1 and 1. 1 being full forward and -1 being full reverse of that thruster.
\begin{figure}[h]
\centering
\includegraphics[width=1\textwidth,left,]{Thruster_Msg.PNG}
\end{figure}

\raggedright
\subsection{Heart Beat Message(RX Only}
\raggedright
Channel = Thruster Channel\\ 
Must be sent by motherboard every 1000 ms or a kill state will occur.
\begin{figure}[h]
\centering
\includegraphics[width=0.7\textwidth,left,]{HeartBeat_Msg.PNG}
\end{figure}

\raggedright
\subsection{Go Check Message(TX Only)}
\raggedright
Channel = Thruster Channel\\ 
Will be trasmitted every 500 ms. Checks the state of the go hall effect. If the hall effect is\\ attached, 'A' for asserted will 
be transmitted.
\raisebox{-4cm}[0pt][0pt]{
\centering
\makebox[\textwidth][c]{\includegraphics[width=0.7\textwidth,left,]{Go_Msg.PNG}}
}

\vspace{3cm}

\pagebreak
\raggedright
\section{Connections}

    NOTE: The designated UART connector was implemented\\
    as an additional relay control pin in the firmware. This is \\
    designed to be a relay to control main power in case of Hard\\
    Kill. The provided hardware does not have a mosfet to accommodate that\\
    so an external one will have to attached. Refer to hardware schematic to\\ 
    determine required external circuit. 


\begin{figure}[h]
\centering
\includegraphics[width=1\textwidth,left,]{KillBoard_connectors.png}
\end{figure}


\end{document}